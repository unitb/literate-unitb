
\documentclass[12pt]{amsart}
\usepackage{geometry} % see geometry.pdf on how to lay out the page. There's lots.
\geometry{a4paper} % or letter or a5paper or ... etc
% \geometry{landscape} % rotated page geometry
\usepackage{unitb}

% See the ``Article customise'' template for come common customisations

\title{A Small Machine}
\author{Simon Hudon}
\date{} % delete this line to display the current date

%%% BEGIN DOCUMENT
\begin{document}

\maketitle

As an example, we look

\begin{machine}{m0}

\newevent{inc}

\begin{invariant}{inv0}
	x = 2 \cdot y
\end{invariant}

\begin{evassignment}{inc}{a0}
	x' = x + 2
\end{evassignment}

\begin{initialization}{a3}
	x = 0
\end{initialization}

%\begin{invariant}{inv1}
%	x
%\end{invariant}

\end{machine}
This is outside a machine's scope

\begin{machine}{m0}

\begin{initialization}{a2}
	y = 0
\end{initialization}

\begin{evassignment}{inc}{a1}
	y' = y + 1
\end{evassignment}

\begin{variable}
	x,y: \Int
\end{variable}

\begin{fschedule}{inc}{f0}
	x = y
\end{fschedule}

\begin{transient}{inc}{tr0}
	x = y
\end{transient}

\begin{invariant}{inv1}
	x \le 10
\end{invariant}

\end{machine}
\end{document}