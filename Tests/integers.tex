
\documentclass[12pt]{amsart}
\usepackage{geometry} % see geometry.pdf on how to lay out the page. There's lots.
\usepackage{unitb}
\usepackage{calculation}
\geometry{a4paper} % or letter or a5paper or ... etc
% \geometry{landscape} % rotated page geometry

% See the ``Article customise'' template for some common customisations

\title{}
\author{}
\date{} % delete this line to display the current date

%%% BEGIN DOCUMENT
\begin{document}

\maketitle
\tableofcontents

%\section{}
%\subsection{}

\begin{machine}{m0}

\hide{
	\begin{variable}
		n,a : \Int
	\end{variable}
}

\begin{invariant}{inv0}
	a = n^3
\end{invariant}

%\begin{proof}{INIT/INV/inv0}
%\begin{calculation}
%	
%\end{calculation}
%\end{proof}

\newevent{evt}

\begin{initialization}{init0}
	n = 0 \land a = 0
\end{initialization}

\begin{evassignment}{evt}{a0}
	n' = n + 1
\end{evassignment}

\begin{proof}{evt/INV/inv0}
	\begin{calculation}
		(n')^3
	\hint{=}{ \ref{a0} }
		(n+1)^3
	\hint{=}{ arithmetic }
		n^3 + 3 \cdot n^2 + 3 \cdot n + 1
	\hint{=}{ \ref{inv0} }
		a + 3 \cdot n^2 + 3 \cdot n + 1
	\hint{=}{ \ref{inv1} }
		a + b
	\hint{=}{ \ref{a1} }
		a'
	\end{calculation}
\end{proof}

\begin{variable}
	b : \Int
\end{variable}

\begin{invariant}{inv1}
	b = 3 \cdot n^2 + 3 \cdot n + 1
\end{invariant}
%\begin{invariant}{inv2}
%	b ~=~ 3 \cdot n^2 + 3 \cdot n + 1
%\end{invariant}

\begin{evassignment}{evt}{a1}
	a' = a + b
\end{evassignment}

\begin{initialization}{in1}
	b = 1
\end{initialization}

\begin{itemize}
\item spacing commands
\item proof structures (proof by cases, etc)
\item label initialization predicates
\item types
\item invariant theorems
\item error checking
\item formatting
\item lazy proof checking
\item continuous checking
\end{itemize}

\begin{proof}{evt/INV/inv1}
	\begin{calculation}
		3 \cdot (n')^2 + 3 \cdot n' + 1
	\hint{=}{ \ref{a0} }
		3 \cdot (n+1)^2 + 3 \cdot (n+1) + 1
	\hint{=}{ arithmetic }
		3 \cdot (n^2+2\cdot n+1) + 3 \cdot (n+1) + 1
	\hint{=}{ arithmetic }
		3 \cdot n^2+6\cdot n+3 + 3 \cdot n+3 + 1
	\hint{=}{ \ref{inv1} }
		b+6\cdot n+3+3
	\hint{=}{ \ref{inv2} }
		b+c
	\hint{=}{ \ref{a2} }
		b'
	\end{calculation}
\end{proof}

\begin{variable}
	c : \Int
\end{variable}

\begin{invariant}{inv2}
	c = 6 \cdot n + 6
\end{invariant}

\begin{evassignment}{evt}{a2}
	b' = b + c
\end{evassignment}

\begin{initialization}{in2}
	c = 6
\end{initialization}

\begin{proof}{evt/INV/inv2}
	\begin{calculation}
		6 \cdot (n') + 6
	\hint{=}{ \ref{a0} }
		6 \cdot (n+1) + 6
	\hint{=}{ arithmetic }
		6 \cdot n + 6 + 6
	\hint{=}{ \ref{inv2} }
		c + 6
	\hint{=}{ \ref{a3} }
		c'
	\end{calculation}
\end{proof}

\begin{evassignment}{evt}{a3}
	c' = c + 6
\end{evassignment}

\end{machine}

\end{document}