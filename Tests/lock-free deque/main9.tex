\documentclass[12pt]{amsart}
\usepackage[margin=0.5in]{geometry} 
  % see geometry.pdf on how to lay out the page. There's lots.
\usepackage{bsymb}
\usepackage{../unitb}
\usepackage{calculational}
\usepackage{ulem}
\usepackage{hyperref}

\newcommand{\REQ}{\text{REQ}}

\begin{document}
  % %!TEX root=../train-station-set.tex
\begin{block}
  \item   \textbf{machine} m0
  \item   \textbf{variables}
  \begin{block}
    \item   $in$
  \end{block}
  \item   \textbf{progress}
\begin{block}
\item[ \eqref{prog0} ]$\true  \quad \mapsto\quad b$ %
\end{block}

  \item   \textbf{transient}
\begin{block}
\item[ \eqref{m0:tr0} ]$t \in in  \qquad \text{(\ref{m0:leave}: [t := t' = t])}$ %
\end{block}

  \item   \textbf{events}
  \begin{block}
    \item   \noindent \ref{m0:enter} [t] \textbf{event}
\begin{block}
\item \textbf{during}
\begin{block}
\item[ \eqref{m0:enterdefault} ]$\false$ %
\end{block}
\item \textbf{begin}
\begin{block}
\item[ \eqref{m0:entera1} ]$in' = in \bunion \{ t \} $ %
\end{block}
\item \textbf{end} \\
\end{block}

    \item   \noindent \ref{m0:leave} [t] \textbf{event}
\begin{block}
  \item   \textbf{during}
  \begin{block}
  \item[ \eqref{m0:leavelv:c0} ]$t \in in $ %
  \end{block}
  \item   \textbf{begin}
  \begin{block}
  \item[ \eqref{m0:leavelv:a0} ]$in \bcmsuch in' = in \setminus \{ t \} $ %
  \end{block}
  \item   \textbf{end} \\
\end{block}

  \end{block}
  \item   \textbf{end} \\
\end{block}

\begin{machine}{m0}

  \newset{\REQ}

We need a variable to keep track of all the requests to mutate the
data structure.

  \[ \variable{ req,req_0 : \set [\REQ] } \]

... and new events:
% % \hide{
  \newevent{req}{req}
  \newevent{handle}{handle}
  % \newevent{req}{req\_b} 
  % \newevent{handle}{handle\_b} 

\[\param{req}{r : \REQ }\]
\with{sets}
\begin{align*}
  & \evguard{req}{grd0}{ \neg r \in req }  \\
  & \evbcmeq{req}{act0}{ req }{ req \bunion \{ r \} } \\
  & \evbcmeq{req}{act1}{ req_0 }{ req } 
\end{align*}
\[\param{handle}{r : \REQ }\]
\begin{align*}
  \evguard{handle}{grd0}{ r \in req }  \\
  \evbcmeq{handle}{act0}{ req }{ req \setminus \{ r \} } \\
  \evbcmeq{handle}{act1}{ req_0 }{ req }
\end{align*}

% \section{Requirements}
%   \dummy{ R : \set[\REQ] }
\begin{align*}
  \constraint{co0}{ req_0' = req \1\lor (req_0' = req_0 \1\land req' = req) }
\end{align*}
\begin{align*}
  & \progress{prog0}
    { \neg req = \emptyset }
    { \neg req_0 \subseteq req }
 \refine{prog0}{ensure}{handle}{ using \ref{handle} }
  & \progress{prog1}
    { V = req}
    { req \subset V 
      \1\lor req = \emptyset \1\lor \neg req \subseteq req_0}
 \refine{prog1}{ensure}{handle}{ using \ref{handle} }
\end{align*}
\begin{align*}
  \dummy{ R,V : \set [\REQ] } \\
  \cschedule{handle}{m0:sch0}{ \neg req = \emptyset } \\
  \initialization{m0:in0}{ req = \emptyset }
\end{align*}
\end{machine}

% %!TEX root=../train-station-set.tex
\begin{block}
  \item   \textbf{machine} m1
  \item   \textbf{variables}
  \begin{block}
    \item   $in$
    \item   $loc$
  \end{block}
  \item   \textbf{invariants}
\begin{block}
\item[ \eqref{inv0} ]$\qforall{p_0,p_1}{}{pc.p_0 \le pc.p_1 + 1} $ %
\end{block}

  \item   %!TEX root=../main9.tex
\textbf{progress}
\begin{block}
\item[ \eqref{m1:prog1} ]{$b = ch \quad \mapsto\quad b = ch$} %
\end{block}

  \item   \textbf{safety}
\begin{block}
\item[ \eqref{saf1} ]$vs = \Pcs \textbf{\quad unless \quad} \false$ %
\item[ \eqref{saf2} ]$\Pcs \setminus vs \subseteq V \textbf{\quad unless \quad} vs = \Pcs$ %
\end{block}

  \item   \textbf{transient}
\begin{block}
\item[ \eqref{m1:tr0} ]$t \in in  %
		\1\land loc.t \in plf  \qquad \text{(\ref{m1:moveout}: [t := t' | t' = t])}$ %
\item[ \eqref{m1:tr1} ]$t \in in \land loc.t = ent  \qquad \text{(\ref{m1:movein}: [t := t' | t' = t])}$ %
\end{block}

  \item   \textbf{events}
  \begin{block}
    \item   \noindent \ref{m0:enter} [t] \textbf{event}
\begin{block}
\item \textbf{during}
\begin{block}
\item[ \eqref{m0:enterdefault} ]$\false$ %
\end{block}
\item \textbf{when}
\begin{block}
\item[ \eqref{m0:enterent:grd1} ]$\neg t \in in $ %
\end{block}
\item \textbf{begin}
\begin{block}
\item[ \eqref{m0:entera1} ]$in' = in \bunion \{ t \} $ %
\item[ \eqref{m0:entera3} ]$loc' = loc \1| t \fun ent $ %
\end{block}
\item \textbf{end} \\
\end{block}

    \item   \input{train-station-set/m1_m0-leave}
    \item   \input{train-station-set/m1_m1-movein}
    \item   \input{train-station-set/m1_m1-moveout}
  \end{block}
  \item   \textbf{end} \\
\end{block}

\begin{machine}{m1}
  \refines{m0}
\[ \indices{handle}{ b : \Bool } \]
\[ \dummy{ b : \Bool } \]
\[ \variable{ ch : \Bool } \]
\begin{align*}
  & \cschedule{handle}{m1:sch0}{ b = ch } \\
  & \witness{handle}{b}{b = ch} \\
  & \progress{m1:prog1}{ b = ch }{ b = ch }
\end{align*}
\replace{handle}{m1:sch0}{m1:prog1}
  \begin{liveness}{m1:prog1}
    \progstep{\true}{req = \emptyset \1\lor \neg req \subseteq req_0}
      {induction}{}{ \var{req}{down}{\emptyset} }
  \begin{flatstep}
    \progstep
      {V = req}
      {req \subset V
        \1\lor req = \emptyset \1\lor \neg req \subseteq req_0}
      {discharge}{}{}
      \begin{step}
        \trstep{handle}{ \index{b}{ \true }  }
          { req = V \1\land \neg req = \emptyset  }
        \safstep
          { V = req }
          { req \subset V \1\lor \neg req \subseteq req_0 }
          {}
      \end{step}
  \end{flatstep}
  \end{liveness}
\end{machine}

% % %!TEX root=../main8.tex
\begin{block}
  \item   \textbf{machine} m2
  \item   \textbf{variables}
  \begin{block}
    \item   $emp$
    \item   $p$
    \item   $ppL$
    \item   $ppR$
    \item   $psL$
    \item   $psR$
    \item   $q$
    \item   $qe$
    \item   $res$
  \end{block}
  \item   \textbf{events}
  \begin{block}
    \item   \input{main8/m2_req-pop-left}
    \item   \input{main8/m2_req-pop-right}
    \item   \input{main8/m2_req-push-left}
    \item   %!TEX root=../main8.tex
\noindent \ref{req:push:right}  \textbf{event}
\begin{block}
  \item   \textbf{during}
  \begin{block}
  \item[ (\ref{req:push:right}/default) ]{$\false $} %
  \end{block}
  \item   \textbf{any} r,x
  \item   \textbf{when}
  \begin{block}
  \item[ \eqref{req:push:rightm0:grd0} ]{$\neg r \in \dom.psR $} %
  \end{block}
  \item   \textbf{begin}
  \begin{block}
  \item[ \eqref{req:push:rightm0:act0} ]{$psR \bcmeq psR \1| r \fun x $} %
  \end{block}
  \item   \textbf{end} \\
\end{block}

    \item   %!TEX root=../main8.tex
\noindent \ref{resp:pop:left} [r] \textbf{event}
\begin{block}
  \item   \textbf{during}
  \begin{block}
  \item[ \eqref{resp:pop:leftm0:sch0} ]{$r \in ppL $} %
  \end{block}
  \item   \textbf{upon}
  \begin{block}
  \item[ \eqref{resp:pop:leftm1:sch0} ]{$\neg p = q$} %
  \end{block}
  \item   \textbf{when}
  \begin{block}
  \item[ \eqref{resp:pop:leftm1:grd0} ]{$\neg p = q$} %
  \end{block}
  \item   \textbf{begin}
  \begin{block}
  \item[ \eqref{resp:pop:leftm0:act0} ]{$ppL \bcmeq ppL \setminus \{ r \} $} %
  \item[ \eqref{resp:pop:leftm1:act0} ]{$p \bcmeq p+1$} %
  \item[ \eqref{resp:pop:leftm1:act1} ]{$qe \bcmeq \{ p \} \domsub qe $} %
  \item[ \eqref{resp:pop:leftm1:act2} ]{$emp \bcmeq \false $} %
  \item[ \eqref{resp:pop:leftm1:act3} ]{$res \bcmeq qe.p $} %
  \end{block}
  \item   \textbf{end} \\
\end{block}

    \item   %!TEX root=../main8.tex
\noindent \ref{resp:pop:left:empty} [r] \textbf{event}
\begin{block}
  \item   \textbf{during}
  \begin{block}
  \item[ \eqref{resp:pop:left:emptym0:sch0} ]{$r \in ppL $} %
  \end{block}
  \item   \textbf{begin}
  \begin{block}
  \item[ \eqref{resp:pop:left:emptym0:act0} ]{$ppL \bcmeq ppL \setminus \{ r \} $} %
  \end{block}
  \item   \textbf{end} \\
\end{block}

    \item   \input{main8/m2_resp-pop-right}
    \item   %!TEX root=../main8.tex
\noindent \ref{resp:pop:right:empty} [r] \textbf{event}
\begin{block}
  \item   \textbf{during}
  \begin{block}
  \item[ \eqref{resp:pop:right:emptym0:sch0} ]{$r \in ppR $} %
  \end{block}
  \item   \textbf{upon}
  \begin{block}
  \item[ \eqref{resp:pop:right:emptym1:sch0} ]{$p = q $} %
  \end{block}
  \item   \textbf{when}
  \begin{block}
  \item[ \eqref{resp:pop:right:emptym1:grd0} ]{$p = q $} %
  \end{block}
  \item   \textbf{begin}
  \begin{block}
  \item[ \eqref{resp:pop:right:emptym0:act0} ]{$ppR \bcmeq ppR \setminus \{ r \} $} %
  \item[ \eqref{resp:pop:right:emptym1:act2} ]{$emp \bcmeq \true $} %
  \end{block}
  \item   \textbf{end} \\
\end{block}

    \item   %!TEX root=../main8.tex
\noindent \ref{resp:push:left} [r] \textbf{event}
\begin{block}
  \item   \textbf{during}
  \begin{block}
  \item[ \eqref{resp:push:leftm0:sch0} ]{$r \in \dom.psL $} %
  \end{block}
  \item   \textbf{when}
  \begin{block}
  \item[ \eqref{resp:push:leftm1:grd0} ]{$r \in \dom.psL $} %
  \end{block}
  \item   \textbf{begin}
  \begin{block}
  \item[ \eqref{resp:push:leftm0:act0} ]{$psL \bcmeq \{r\} \domsub psL $} %
  \item[ \eqref{resp:push:leftm1:act0} ]{$p \bcmeq p-1$} %
  \item[ \eqref{resp:push:leftm1:act1} ]{$qe \bcmeq qe \1| p\0-1 \fun psL.r$} %
  \end{block}
  \item   \textbf{end} \\
\end{block}

    \item   \input{main8/m2_resp-push-right}
  \end{block}
  \item   \textbf{end} \\
\end{block}

% \begin{machine}{m2}
%   \refines{m0}

% We now partitial $req$ into requests for operation A ($reqA$) and
% requests for operation B ($reqB$).

% \[ \variable{ reqA, reqB : \set [\REQ] } \]
% \begin{align*}
%   \invariant{m1:inv0}{ reqA \bunion reqB = req } \\
%   \invariant{m1:inv1}{ reqA \binter reqB = req } 
% \end{align*}
% And consequ
% \begin{align*}
%   \initialization{m1:in0}{ reqA = \emptyset \land reqB = \emptyset }
% \end{align*}
% \end{machine}

\end{document}
